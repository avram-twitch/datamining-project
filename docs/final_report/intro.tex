Spanish artist Pablo Picasso allegedly once said "Good artists copy, great artists steal."
The quote becomes all the more shocking considering that Picasso is one of the most
well-regarded artists ever, and invented an entirely new style of painting.
But artists across all mediums--painters, musicians, writers--
do not create their work in a vacuum.
They get inspired by, collaborate with, respond to, and even copy or steal from other artists.
Artists are often frank in their inspirations as well, citing other artists as influences.

The motivation of our project is to explore how artists influence each other.
We are doing this in the context of contemporary music.
In music, there are commonly accepted narratives as to how certain artists or genres
influence each other.
Human listeners can often identify an artist's influences by common
characteristics between the artists, such as instrumentation, tempo, lyrics, song structure,
or beats.

We want to see if we can identify and quantify these influences and similarities
between artists, genres, and years.
Specifically, we want to see if we can identify "progenitor" songs or artists
that had a great impact on future songs.
The idea is that, for example with Rock music, presumably an artist such as The Beatles
will have an outsized impact on subsequent Rock artists.
If we consider The Beatles as a "progenitor" group,
can we use characteristics from their songs to link to those influenced by their music?
Or, put in another way, can we identify artists who "steal" from The Beatles?
