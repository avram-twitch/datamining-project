It's hard to put a lot of work into an idea that generates no results.
However, such a result does provide a lot of room for reflection, learning, and considerations on how to move forward.
We discuss below some of the broader ideas we learned.

\subsection{Going Forward}

We showed that, as is, our extension of the Lloyds algorithm is ineffective at identifying musical "progenitors."
But while our current approach yielded no results,
we have ideas for how we could potentially get our method to work in the future.

One issue is that we select the oldest songs in a cluster at each step.
There is no guarantee that any intermediate cluster in the algorithm is well separated,
and yet the oldest songs are treated as centers.
Perhaps a better approach would be to run a normal Lloyd's algorithm,
and only after those clusters are determined do we select the oldest among the clusters,
and then repeat the Lloyds algorithm again, until convergence.

Another issue, as mentioned above in the methods section,
is how we define the "oldest" songs in a cluster.
Tweaking that parameter would change how the centers are decided,
and could potentially yield better results.

Lastly, we could use different variables or features.
We used every segmented variable available to us, but we left out the summarized variables.
With the high dimensionality of the segmented variables, this may not impact it too much.


\subsection{Data Cleaning Takes Time}

At least in our case, the most difficult part of our project was not implementing and extending an algorithm
from class, but in actually working with the data.
We spent a large chunk of our time, thought,
and energy on just getting the data into a format that we could use.
We had to consider how to deal with an unfamiliar format,
what form the data needed to be in to be able to use clustering,
how to deal with data of varying dimensions and length,
how to simplify and summarize certain types of data,
and how to process everything in an efficient manner.
Fortunately, our data was already fairly clean to begin with,
but even then it was a difficult task to clean.

Another aspect of data cleaning we didn't really consider until towards the end
was how to work with our own results.
Just having the data clustered isn't enough; we have to draw conclusions from those clusters.
We also spent a lot of time getting the results in a form that we could use to plot and analyze.

\subsection{Final Thoughts}

Working on an extension of a data mining method was a thought-provoking and intriguing project.
We had to consider properties of the Lloyds method,
and how our extension could potentially impact those properties.
We also had to consider how applying our method would work with our particular dataset.
In the end, it was a lot of work, but a lot more rewarding to apply to a project like this.
