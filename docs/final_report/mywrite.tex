    
\documentclass[12pt]{article}

\usepackage{graphicx}
\usepackage[margin=1.5in]{geometry}
\begin{document}
\subsection*{Expectation of the clustering}
\begin{itemize}
	\item We are attempting to cluster all songs around the characteristics of the older songs. Here, we are using the counts of songs in each decade to see if there are more songs in one cluster than the other.
	\item We are expecting some clusters to only have newer songs or older songs. Thus, this way we know that there has been a decade that has not influenced any upcoming decades.
	\item  We are expecting the clusters to have most of the certain decades in just one cluster. For example: most of 1950's and 1970's songs to be in one cluster. Thus, this way we know that there has been a decade that has  influenced coming decade.
	\item As we are using k-grams to cluster the songs attributes, we expect some of the the k-grams to do better than other. 
	\item We are also testing the k-clusters to determine which worked the best. 
\end{itemize}
\subsection*{Results}
\begin{itemize}
\item We wanted to visualize the clusters such that it would provide us with the something that is meaningful and is concise. We decided to visualize the songs in each cluster by decades and their counts. The meaningful way to display this would be with stacked bar chart.
\item Referencing the stacked bar chart below, we can see that most of the songs in the data set are newer songs. 
\item We can see that all of the bars have almost equal proportionate amount of songs based on the decades.
\item No cluster have only one set of data in it. But we can tell that the 2000's have certainly influenced 2010's. However, the songs have been distributed among the clusters so that might not be 100\% true.
\item Using Various Grams didn't really affect the chart much. The charts show that even with various k-grams, the songs have been distributed in similar fashion.  
\end{itemize}

\begin{center}
\includegraphics[scale=0.28]{assests/k3n4.png}
\includegraphics[scale=0.28]{assests/k3n5.png}
\includegraphics[scale=0.28]{assests/k3n6.png}

\includegraphics[scale=0.28]{assests/k4n4.png}
\includegraphics[scale=0.28]{assests/k4n5.png}
\includegraphics[scale=0.28]{assests/k4n6.png}

\includegraphics[scale=0.28]{assests/k5n4.png}
\includegraphics[scale=0.28]{assests/k5n5.png}
\includegraphics[scale=0.28]{assests/k5n6.png} \\
\textit{Fig 1 - Bar Charts showing the counts of songs in each clusters per decade, using various k-grams and k-clusters}
\end{center}



\subsection*{What did I Learn}
\begin{itemize}
\item All of the songs in the dataset didn't have date. This could have influenced some of the results we see in the charts. We are not seeing much of the older songs in the chart is because of this reason.
\item Llyod's Algorithm works well with large dataset and is very scalable. It doesn't require the extensive computation of distances compared to Hierachical.
\item Maybe our approach of basing the center around the average of old songs doesn't really cluster the songs as we expected it to. Maybe we could have tried running with normal Lloyd's clustering  where each cluster's center as the average of all data points in the cluster.
\item The structure we hoped to find wasn't there at all or maybe our approach of finding it was insufficient.
\end{itemize}

\end{document}
