\subsection{Lloyd's Extension}

To identify similarities between songs, we extended the Lloyd's clustering algorithm.
Since we are focusing on finding progenitors, we wanted to adapt Lloyd's algorithm
in a way that would place cluster centers around the oldest songs in the dataset.
That way, we could attempt cluster all songs--both old and newer songs--
around characteristics of older songs.

Normal Lloyd's clustering sets each cluster's center as the average of all data points in the cluster.
Our extension only looks at the oldest songs (by release year) in each cluster,
and sets the center as the average of those oldest songs.

One parameter to consider with our clustering extension is how do we define the "oldest" songs.
For example, we could consider it as the oldest 1, 5, or 10\% of the songs in a cluster.
However, for this report, we defined this as the earliest year of release in a cluster.
All songs that were released that year in the cluster are averaged to be set as the center.

\subsection{Expectations}

As far as expectations for results, we hope to see good separation between genres, and perhaps between artists as well.
This is because if our "musical lineage" hypothesis is correct,
we would expect to see very distinct genres in separate clusters with their "ancestors".
We don't expect to see good separation across years or decades,
because presumably the influence of progenitors will be present throughout the decades.
